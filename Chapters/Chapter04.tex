%************************************************
\chapter{Conclusiones y Perspectivas}\label{ch:conclusiones}
%************************************************

% While it has not been possible to provide definite answers to the questions (Unsuccessful experiment, but I still hope to get it published)
% Three of the samples were chosen for detailed study (The other results didn't make any sense)
% A highly significant area for exploratory study (A totally useless topic selected by my commitee)
% In my experience / in case after case / in a series of cases
% Correct within an order of magnitude / Can be regarded good within some range of validity
% It is clear that much additional work will be required before a complete understanding of this phenomenon occurs
% It is hoped that this study will stimulate further investigations in this field

Este capítulo presenta algunas conclusiones, y al final menciona ideas que podrían ser de utilidad para extender y refinar la investigación iniciada en este trabajo. 

\section{Conclusiones}

%\paragraph {La transformación de modelos discretos en semicontinuos (ecuaciones de Glass) basada en estimación de parámetros es posible}Se mostró que es posible la transformación de modelos discretos en semicontinuos mediante un ejemplo consistente en una pequeña red Booleana sincronizada compuesta por tres nodos. En este caso solo fue necesario estimar tres parámetros y se contó con un solo tipo de medición ``experimental'' o señal original contra la cual comparar. Para este tipo de problema, la elección de estrategia de búsqueda no tiene mayor consecuencia, ya que es posible obtener resultados satisfactorios aún con la búsqueda aleatoria. El índice de pendiente \textsc{(si)} mostró ser una buena medida de comparación entre señales.  

\paragraph {La transformación de un modelo discreto en semicontinuo produce buenos resultados en sistemas donde la relación \emph{parámetros a estimar}/\emph{tipos de mediciones} es alta}El modelo Booleano de 3 nodos muestra que el paso de un formalismo de modelación a otro puede ser exitoso cuando se cuenta con datos suficientes contra los cuales comparar el modelo, en relación a la cantidad de parámetros a estimar. Por el contrario, en modelos como el de la vía señalización, en los que la cantidad de parámetros es mucho mayor que los criterios contra los cuales comparar, es difícil encontrar resultados biológicamente relevantes.

\paragraph {La elección de parámetros para modelos de Glass no es simple} A pesar del éxito que han tenido algunos modelos (\citeauthor{AlvarezBuylla:2008cg} \citep{AlvarezBuylla:2008cg}, \citeauthor{Azpeitia:2010ik} \citep{Azpeitia:2010ik}) en reproducir el comportamiento observado experimentalmente, este trabajo muestra que no siempre es directo encontrar un conjunto de parámetros tales que la dinámica del modelo tenga una interpretación biológica correcta. Establecer los inversos de tiempo característicos en un valor de $1$ y los umbrales de cada nodo en $0.5$ parece ser una buena primera elección ante falta de información detallada, pero no necesariamente es correcta en todos los casos. En los dos trabajos mencionados arriba se estableció el valor de los umbrales en $0.5$ tras hacer un estudio de robustez, pero la comparación se hace cuando el sistema ya ha llegado a un estado estacionario.

\paragraph {Hacer estimación de parámetros para fenómenos no estacionarios usando modelos que asumen estacionariedad es complicado} Hacer estimación de parámetros en modelos de redes de regulación genética o en redes neuronales artificiales puede dar buenos resultados si el estado estacionario del modelo se compara con mediciones experimentales estacionarias. Si en cambio, se compara un modelo estacionario con mediciones no estacionarias, la estimación de parámetros es más difícil y puede ser que no se logre encontrar un modelo que explique las características del sistema. Este es el caso del modelo de ecuaciones de Glass para la vía de señalización presentado en este trabajo: el modelo discreto y el de Glass asumen que se llegará a un estado estacionario y que es ahí donde se hará la estimación de parámetros. Sin embargo, las mediciones experimentales con las que se pretende determinar si el modelo es correcto o no, no son estacionarias. Esa no estacionariedad no permite que los criterios propuestos para comparar las series temporales de modelo y experimento sean suficientes para discriminar de entre un conjunto grande de modelos posibles.

\section{Perspectivas y trabajo futuro}

Este trabajo se desarrolló bajo el supuesto de que la transformación de modelos discretos en semicontinuos podía realizarse mediante el empleo de técnicas de estimación de parámetros, particularmente aquellas basadas en estrategias evolutivas. Las funciones objetivo utilizadas se basan en criterios simples de correlación o minimización de diferencias. 

Existe un marco teórico más desarrollado para la solución de problemas inversos, que cuenta con técnicas elaboradas para el problema de discriminación de modelos. Una técnica que puede resultar prometedora para la solución de este tipo de problemas puede ser la \textsc{Regularización Promotora de Dispersión o Sparsity Enforcing Regularization}, \citeauthor{Engl2009} \citep{Engl2009}, en donde se busca hacer uso de criterios estadísticos para la comparación entre mediciones experimentales y modelos.

Otra posibilidad que valdría la pena explorar consiste en modificar el modelo discreto de modo que contenga menos nodos y por lo tanto haya menos parámetros que estimar. Hasta este momento cuál es un buen criterio general de reducción de modelos discretos sigue siendo una pregunta abierta. Sin embargo, se ha explorado la posibilidad de utilizar criterios de robustez de las redes para separar nodos esenciales para mantener la dinámica y función de la red de aquellos nodos que podrían parecer redundantes.

Usando este último criterio, se ha logrado plantear un modelo discreto de la vía de señalización de speract con 11 nodos. Queda abierto el determinar si a partir de este modelo reducido y los datos experimentales con que se cuenta hasta el momento, es posible ajustar el comportamiento de un modelo semicontinuo de manera que la interpretación biológica del modelo sea más completa que lo alcanzado hasta el momento.

%Una alternativa a otro tipo de comparaciones es la \textsc{Regularización Promotora de Dispersión o Sparsity Promoting Regularization}, \citeauthor{Engl2009}. La idea consiste en imputar un modelo estadístico a la comparación de mediciones experimentales con modelos de ecuaciones diferenciales. Los operadores diferenciales que no incluyen componentes estocásticos suelen producir trayectorias por lo general suaves, en contraste con la variabilidad existente en las series de tiempo de las mediciones experimentales. 
%
%Sea $\mathbf{p}$ el vector de parámetros de un conjunto de ecuaciones diferenciales. Sea $\Phi(\mathbf{p},t)$ la trayectoria solución dependiente del tiempo $t$ y del vector de parámetros $\mathbf{p}$. Sea $\Psi(\Phi(\mathbf{p},t))$ un modelo Gaussiano del ruido de las mediciones experimentales con desviación estándar constante. El objetivo de añadir $\Phi$  es imputar un criterio estadístico a la trayectoria solución de la ecuación diferencial, y hacer entonces una comparación entre un modelo estadísticos y un conjunto de datos.

% It is hoped that this study will stimulate further investigations in this field

