%*******************************************************
% Abstract
%*******************************************************
%\renewcommand{\abstractname}{Abstract}
\pdfbookmark[1]{Resumen}{Resumen}
\begingroup
\let\clearpage\relax
\let\cleardoublepage\relax
\let\cleardoublepage\relax

\chapter*{Resumen}

Se presenta un modelo semicontinuo para una vía de señalización presente en el flagelo del espermatozoide de erizo de mar, que ha sido relacionada con la regulación de la motilidad del espermatozoide. El modelo semicontinuo fue construído a partir de un modelo discreto propuesto previamente por \citeauthor{Espinal2011} \citep{Espinal2011}, con el objetivo de realizar comparaciones cuantitativas entre comportamientos \emph{in silico} e \emph{in vivo}, a diferencia del modelo anterior, donde las comparaciones son a nivel cualitativo. Debido a la mayor complejidad de las reglas lógicas del modelo discreto antes mencionado en comparación con otros modelos discretos desarrollados para redes de regulación genética, se muestra un caso de estudio simple basado en la transformación de un modelo discreto compuesto de pocos nodos en uno basado en el formalismo semicontinuo aquí presentado. Si bien hay estudios que han sido exitosos en construir modelos semicontinuos a partir de modelos discretos, en el presente trabajo se requiere de una elección de parámetros más compleja para la cual es necesario contar con herramientas más elaboradas. A pesar de que en esta tesis no se logra establecer una metodología adecuada para esta transformación, se exploran pasos en esta dirección.

\section*{Organización de la tesis}

El capítulo \ref{ch:antecedentes} introduce la vía de señalización relacionada con la motilidad del espermatozoide de erizo de mar, un formalismo discreto para la modelación de vías de señalización, así como un modelo desarrollado bajo este formalismo por \citeauthor{Espinal2011} \citep{Espinal2011}. Tras discutir sus alcances y limitaciones, se introduce un formalismo semicontinuo que permite retomar modelos discretos y transformar el estado de los componentes de la red y el tiempo en variables continuas. Se discuten también algunos aspectos necesarios para la construcción de modelos basados en el formalismo semicontinuo. Finalmente, se presentan la justificación y objetivos del presente trabajo.

El capítulo \ref{ch:matmet} presenta los datos y técnicas usadas para la transformación de los modelos discretos en semicontinuos.

El capítulo \ref{ch:resultados} muestra los modelos semicontinuos que se obtuvieron a partir de modelos discretos y discute su validez de acuerdo a los comportamientos observados y el conocimiento biológico previo.

El capítulo \ref{ch:conclusiones} contiene las conclusiones del presente trabajo y perspectivas de trabajo a futuro.

El apéndice \ref{ch:3Nodos} describe el modelo de tres nodos usado como caso de estudio simple.

Finalmente, se presenta la bibliografía consultada para este proyecto.

%En los organismos sexuados, la fecundación es un proceso fundamental para la preservación de la vida. En este evento es necesario que el espermatozoide nade en busca del óvulo. Uno de los organismos modelo que se ha utilizado para estudiar este proceso es el erizo de mar, que produce una gran cantidad de espermatozoides en cada eyaculación y cuyo proceso de feundación es externo.

%En este organismo modelo, varios resultados experimentales han relacionado distintos patrones de nado con una vía de señalización bioquímica que, al ser excitada por una molécula llamada speract que es liberada por el óvulo, induce un proceso de polarización y depolarización de la membrana celular del flagelo del espermatozoide. Este proceso de polarización y depolarización se debe a la entrada y salida de iones a través de distintos canales iónicos situados en la membrana. A través de un marcador fluorescente, es posible medir experimentalmente la concentración intracelular de uno de estos iones, el calcio.

%En su estado nativo, el espermatozoide nada en modo circular. En la cercanía del óvulo, el espermatozoide se ve expuesto a un gradiente de speract, lo cual activa la vía de señalización antes mencionada. Como consecuencia de esta activación, el espermatozoide hace un giro brusco, seguido de un pequeño período en el que parece nadar de manera recta para luego intentar recuperar su nado circular. Cada molécula de speract que logra pegarse al receptor específico en la membrana del flagelo induce nuevamente esta vía, por lo cual se puede observar una alternancia de giros bruscos y nado recto. 

%En una especie de erizo, este proceso parece ser guiado conforme el espermatozoide nada hacia el lugar donde la concentración de speract es mayor, es decir, en dirección hacia el óvulo. En este sentido se puede decir que el speract funciona como un quimioatractor para el espermatozoide. Sin embargo, en otra especie de erizo el espermatozoide presenta los giros bruscos causados por el speract, pero no parece ser atraído hacia el óvulo. Si bien diferentes resultados experimentales han sido de utilidad para establecer cuáles componentes bioquímicos toman parte en la vía de señalización y por ende de la maquinaria de control de motilidad del espermatozoide, aún quedan preguntas por responder con respecto al papel que juega cada uno de dichos componentes en la dinámica de la via de señalización. 

%En este sistema biológico, probar diferentes estados y condiciones de manera experimental requiere de metodologías muy elaboradas. Sin embargo, también es posible utilizar distintas clases de modelos computacionales, con los cuales se simule una gran cantidad de condiciones y estados.

%Se pueden crear distintos tipos de modelos dinámicos de acuerdo a diferentes criterios y formalismos, que van desde aquellos en los que no se considera el espacio, el tiempo y estado de los componentes del sistema son descritos de manera discreta; a aquellos en los que espacio, tiempo y estado son variables continuas. Así mismo, hay modelos que incorporan variables estocásticas para describir el comportamiento del sistema a través del tiempo. Cada formalismo requiere de una cantidad y calidad distinta de datos.

%Entre los modelos que requieren una menor cantidad de datos se encuentran las redes lógicas, donde tiempo y estado son discretos, y cada componente del sistema actualiza su valor en el tiempo de acuerdo a una regla de evolución discreta. A pesar de su simplicidad, estos modelos han demostrado una gran capacidad de recuperar de manera cualitativa el comportamiento de un sistema, a la par que son capaces de generar predicciones nuevas que sugieran comportamientos susceptibles de ser verificados de manera experimental.

%En particular, el artículo de \citet{Espinal2011} presenta un modelo discreto en tiempo y estado para la vía de señalización antes mencionada. Este modelo logró reproducir observaciones experimentales a la par de sugerir la existencia de comportamientos que no se habían estudiado anteriormente, y que fueron corroborados al realizar experimentos bajo las condiciones señaladas por la nueva predicción.

%A pesar del éxito obtenido por este modelo, el tipo de comparaciones que pueden hacerse con las mediciones experimentales se mantiene a un nivel cualitativo. Para poder realizar comparaciones cuantitativas, se requiere que tiempo, estado o ambos sean variables continuas. Una manera de resolver este problema es construir un modelo consistente en un conjunto de ecuaciones diferenciales ordinarias \textsc{EDOs} que reproduzcan, al igual que el modelo discreto, las observaciones experimentales. Sin embargo, los modelos basados en \textsc{EDOs} requieren de un conocimiento más detallado de las interacciones, así como de las concentraciones de los distintos componentes de una vía de señalización. En particular para esta vía de señalización, muchas de estas cantidades no son conocidas, y medirlas experimentalmente es un proceso complicado y en muchas ocasiones costoso. 

%Una alternativa es retomar el modelo discreto y transformarlo al formalismo de las ecuaciones de Glass. Este tipo de ecuaciones son de tipo diferencial lineal por pedazos, es decir, la dinámica del sistema se divide en intervalos de tiempo muy pequeños, y se definen sendas ecuaciones diferenciales lineales. La forma específica que toma la ecuación diferencial depende del valor de la función discreta sobre la cual se construyó dicha ecuación.

%Las ecuaciones de Glass permiten hacer una comparación más directa con mediciones y condiciones experimentales. A pesar de que su formulación es muy sencilla, requieren de la estimación de un conjunto de parámetros, relacionados con el tiempo característico de reacción de cada componente y parámetros de umbral. Estos umbrales permiten discretizar mediante funciones escalón las variables de estado continuas, de modo que puedan ser evaluadas adecuadamente por las funciones discretas y pueda obtenerse una forma concreta de ecuación diferencial en cada intervalo de tiempo.

%Elegir adecuadamente un conjunto de parámetros tales que reproduzcan las mediciones experimentales de calcio y concuerden con el conocimiento biológico de los distintos componentes de la vía de señalización, no es un problema trivial. Sin embargo, la estimación de parámetros puede ser expresada como un problema de optimización, en la que se busque minimizar la diferencia entre dos trayectorias a lo largo del tiempo.

%El hecho de plantear un problema de optimización requiere del uso de estrategias de exploración del espacio de soluciones de la función objetivo, especialmente cuando no se tiene una idea clara de los gradientes de la función objetivo. Algunas estrategias de exploración incluyen búsqueda aleatoria, algoritmos genéticos y evolución diferencial. Estos dos últimos son estrategias evolutivas que han demostrado su utilidad en una gran variedad de situaciones para las cuales el paisaje de la función objetivo no es necesariamente diferenciable y en general para cuando este no es conocido ampliamente.

%Como una primera aproximación al problema de transformar un sistema discreto en uno semicontinuo, se consideró un modelo de red Booleana de tres nodos que se transformó en un sistema de ecuaciones de Glass sincronizado. Una de las soluciones fue considerada como señal experimental y se puso en marcha la estimación de los parámetros que generaron dicha señal. En este caso se pudieron recuperar los parámetros que generaron dicha señal.

%En el caso del modelo para la red de señalización, se usó la noción de distancia para comparar las mediciones experimentales de calcio con la dinámica de Glass del nodo que representa al calcio, y se presentan los resultados obtenidos.

%Esta tesis se organiza como sigue: en el capítulo \ref{ch:antecedentes} se discute brevemente el fenómeno biológico que se quiere entender a través de modelación computacional. Posteriormente se discuten un modelo de red Booleana de tres nodos que se usó para ganar entendimiento de cómo transformar un modelo discreto en uno semicontinuo, así como del modelo de la vía de señalización basado en funciones discretas. El capítulo \ref{ch:matmet} presenta una discusión de los alcances y limitaciones del modelo discreto a manera de motivación y justificación para el desarrollo de este trabajo; posteriormente introduce el formalismo de ecuaciones de Glass; plantea el problema de la estimación de parámetros como un problema de optimización, describiendo las nociones de distancia usadas para comparar la dinámica de calcio de Glass con mediciones experimentales; finalmente aborda brevemente el procedimiento seguido para poner en marcha la búsqueda de parámetros.
%\ref{ch:resultados} presenta los resultados de la búsqueda de parámetros. Posteriormente se presentan conclusiones y trabajo a futuro en \ref{ch:conclusiones}. Finalmente se presenta la bibliografía consultada para este proyecto.

\endgroup			

\vfill