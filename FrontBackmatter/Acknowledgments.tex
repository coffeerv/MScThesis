%*******************************************************
% Acknowledgments
%*******************************************************
\pdfbookmark[1]{Acknowledgments}{acknowledgments}

\begin{flushright}{\slshape
	Dente lupus, cornu taurus petit}\\
\medskip
    --- Popular.
\end{flushright}

\begin{flushright}{\slshape
	Draco dormiens nunquam titillandus}\\
\medskip
    --- J. K. R.
\end{flushright}


\bigskip

\begingroup
\let\clearpage\relax
\let\cleardoublepage\relax
\let\cleardoublepage\relax
\chapter*{Agradecimientos}
Este trabajo representa no solo largas horas frente a una computadora, sino tal vez más horas junto a una taza de café, un pizarrón, bibliotecas, jardines, congresos, sesiones de carteles, y una larga lista que queda obviada en un etc.

A diferencia de las horas frente a la computadora, que en algún sentido resultan más solitarias e individuales, el resto del tiempo invertido (o dejado de invertir) en la realización de este trabajo es un esfuerzo colectivo de muchas personas que me dieron su apoyo en muchos y distintos momentos, ya sea dándome ánimos, ideas, escuchando, criticando, proponiendo.

Agradezco a mis padres, cuyo apoyo y cariño me ha permitido proponerme retos y lograrlos. A Libertad, que me muestra que el mundo tiene muchos colores, que encuentra mil y un maneras de impulsarme y siempre tiene algo qué compartir para ponerme feliz =). 
Por supuesto, a mis amigos en el ICF y CCG, con quienes he podido no solo ñoñear sino pasar muchos ratos agradables: Adriana, Alex Franco, Alex Priego, Alfredo, Ana Virginia, Anna Contrerova, Ángel, (El) Arlets, Aurora, Bertha, Christian, Chucho, Claudia, Cristina, Gloria, Jared, José Luis, Karel, Lina, Luisana, Mayra, Mauricio, Ricardo, Roberto, Santiago, Vanessa, Yalbi. Mención especial al Colectivo Cubil Felino.

Gracias también a Gustavo, Max, Hernán, José, Markus, Daniel, Raúl. Sus consejos han sido de mucha utilidad, pero sobre todo, su amistad ha sido y es muy apreciada. Al Dr. Marcos Capistrán por recibirme brevemente en \textsc{cimat} y aportar valiosas ideas.

La Facultad de Ciencias de la UAEM y el Instituto de Ciencias Físicas de la UNAM proveyeron un lugar agradable y adecuado en el cual realizar este trabajo. Al último pero no menos importante, mención especial a la cafetera del ICF.

\bigskip

Este trabajo cont\'o con una beca de maestr\'ia por parte de \textsc{conacyt} con n\'umero 241439.
Adem\'as, se cont\'o con el apoyo del proyecto \textsc{papiit in-109210} de la \textsc{dgapa-unam}.
\endgroup



